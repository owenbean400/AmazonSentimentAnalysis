%
% File ranlp2023.tex
%
%% Based on the style files for ACL-IJCNLP 2021, which were
%% Based on the style files for EMNLP 2020, which were
%% Based on the style files for ACL 2020, which were
%% Based on the style files for ACL 2018, NAACL 2018/19, which were
%% Based on the style files for ACL-2015, with some improvements
%%  taken from the NAACL-2016 style
%% Based on the style files for ACL-2014, which were, in turn,
%% based on ACL-2013, ACL-2012, ACL-2011, ACL-2010, ACL-IJCNLP-2009,
%% EACL-2009, IJCNLP-2008...
%% Based on the style files for EACL 2006 by 
%%e.agirre@ehu.es or Sergi.Balari@uab.es
%% and that of ACL 08 by Joakim Nivre and Noah Smith

\documentclass[11pt,a4paper]{article}
\usepackage[hyperref]{project}
\usepackage{times}
\usepackage{latexsym}
\renewcommand{\UrlFont}{\ttfamily\small}

% This is not strictly necessary, and may be commented out,
% but it will improve the layout of the manuscript,
% and will typically save some space.
\usepackage{microtype}

%\aclfinalcopy % Uncomment this line for the final submission
%\def\aclpaperid{***} %  Enter the acl Paper ID here

%\setlength\titlebox{5cm}
% You can expand the titlebox if you need extra space
% to show all the authors. Please do not make the titlebox
% smaller than 5cm (the original size); we will check this
% in the camera-ready version and ask you to change it back.

\title{Affects of Proper Nouns on Amazon Sentiment Analysis}

\author{Anne Turmel \\
  USM / Address line 1 \\
  Affiliation / Address line 2 \\
  Affiliation / Address line 3 \\
  \texttt{anne.turmel@maine.edi} \\\And
  Owen Bean \\
  Affiliation / Address line 1 \\
  Affiliation / Address line 2 \\
  Affiliation / Address line 3 \\
  \texttt{owen.bean@maine.edu} \\}

\date{}

\begin{document}
\maketitle
\begin{abstract}
  % TO DO
Abstraction needs to be written after writing everything else
\end{abstract}

\section{Introduction}

% adjectives, adverbs, nouns, and verbs can obtain sentiment detail/ - Page 11 of 
% nouns are important for content details - Page 19
% Object is a noun, proper noun, and sometimes a verb - Page 25

Sentiment Analysis looks into extracting subjective information from a source. It has been noted from Bing Liu sentiment analysis on subjectivity that adjectives, adverbs, nouns, and verbs can obtain sentiment detail \cite{subject}. Furthermore, an object in object extraction from sentiment is a noun, proper noun, and sometimes a verb. Even though nouns are a way of obtaining subjective information, further investigation has not been looked into for proper nouns.

There has been plenty of models on predicting sentiment analysis of Amazon reviews. The top 3 models; accurancies unsupervised data augmentation for consistency training \cite{unsupervised}, deep pyramid convolutional neural networks for text categorization \cite{pyramid}, and disconnected recurrent neural networks for text categorization \cite{disconnect}; all has been tested with Amazon product reviews from users. Although those models accurancies are between 60\%-70\% percent, does simplifing the proper noun into one word affect Amazon sentiment analysis in positive or negative accurancies.

\section{Related Work}

\section{Task}

The task is to see if proper nouns contains necessary information needed for sentiment analysis in Amazon product reviews.

\section{Method}

Amazon product review data comes with a rating and review that can be used for training. The product must parse out proper nouns with a replacement token to represent their is a proper noun, however, removes the context of the proper noun. An example of removing the proper noun context can be shown.

\begin{small}
\begingroup\makeatletter\def\@currenvir{verbatim}
\verbatim
"It's just like having $100. Except I
couldn't get Walmart to accept it.
Apparently you can only use it on the
interweb at Amazon." 

Detect Amazon and Walmart as proper noun
replaced with <NNP>

"It's just like having $100. Except I
couldn't get <NNP> to accept it. 
Apparently you can only use it on 
the interweb at <NNP>."
\end{verbatim}
\end{small}

% Picture representation
% "It's just like having $100. Except I couldn't get Walmart to accept it. Apparently you can only use it on the interweb at Amazon." -> Detect Amazon and Walmart as proper noun -> "It's just like having $100. Except I couldn't get <NNP> to accept it. Apparently you can only use it on the interweb at <NNP>."

\subsection{LUKE}

LUKE is an entity extraction model that can be used for extracting proper nouns in an Amazon product review.

\subsection{NLTK Speech Tagging}

NLTK library comes with speech tagging library that can be used for tagging the proper nouns in a text.

\subsection{Unsupervised Data Augmentation}

Unsupervised data augmentation is a way to train the model. Can make comparsion on data with proper noun context and without.

\subsection{Deep Pyramid Convolutional Neural Networks for Text Categorization}

Deep Pyramid Convolutional Neural Networks for Text Categorization is another method for training a model for sentiment analysis. Can make comparsion on data with proper noun context and without.

\subsection{FastText}

FastText can be used in a bag of tricks for efficient text classifcation. Can make comparsion on data with proper noun context and without.

\section{Experiments}

\section{Conclusion}

\bibliographystyle{plain}
\bibliography{project}

\end{document}